% Options for packages loaded elsewhere
% Options for packages loaded elsewhere
\PassOptionsToPackage{unicode}{hyperref}
\PassOptionsToPackage{hyphens}{url}
\PassOptionsToPackage{dvipsnames,svgnames,x11names}{xcolor}
%
\documentclass[
  portuguese,
  letterpaper,
  DIV=11,
  numbers=noendperiod]{scrreport}
\usepackage{xcolor}
\usepackage{amsmath,amssymb}
\setcounter{secnumdepth}{5}
\usepackage{iftex}
\ifPDFTeX
  \usepackage[T1]{fontenc}
  \usepackage[utf8]{inputenc}
  \usepackage{textcomp} % provide euro and other symbols
\else % if luatex or xetex
  \usepackage{unicode-math} % this also loads fontspec
  \defaultfontfeatures{Scale=MatchLowercase}
  \defaultfontfeatures[\rmfamily]{Ligatures=TeX,Scale=1}
\fi
\usepackage{lmodern}
\ifPDFTeX\else
  % xetex/luatex font selection
\fi
% Use upquote if available, for straight quotes in verbatim environments
\IfFileExists{upquote.sty}{\usepackage{upquote}}{}
\IfFileExists{microtype.sty}{% use microtype if available
  \usepackage[]{microtype}
  \UseMicrotypeSet[protrusion]{basicmath} % disable protrusion for tt fonts
}{}
\makeatletter
\@ifundefined{KOMAClassName}{% if non-KOMA class
  \IfFileExists{parskip.sty}{%
    \usepackage{parskip}
  }{% else
    \setlength{\parindent}{0pt}
    \setlength{\parskip}{6pt plus 2pt minus 1pt}}
}{% if KOMA class
  \KOMAoptions{parskip=half}}
\makeatother
% Make \paragraph and \subparagraph free-standing
\makeatletter
\ifx\paragraph\undefined\else
  \let\oldparagraph\paragraph
  \renewcommand{\paragraph}{
    \@ifstar
      \xxxParagraphStar
      \xxxParagraphNoStar
  }
  \newcommand{\xxxParagraphStar}[1]{\oldparagraph*{#1}\mbox{}}
  \newcommand{\xxxParagraphNoStar}[1]{\oldparagraph{#1}\mbox{}}
\fi
\ifx\subparagraph\undefined\else
  \let\oldsubparagraph\subparagraph
  \renewcommand{\subparagraph}{
    \@ifstar
      \xxxSubParagraphStar
      \xxxSubParagraphNoStar
  }
  \newcommand{\xxxSubParagraphStar}[1]{\oldsubparagraph*{#1}\mbox{}}
  \newcommand{\xxxSubParagraphNoStar}[1]{\oldsubparagraph{#1}\mbox{}}
\fi
\makeatother


\usepackage{longtable,booktabs,array}
\usepackage{calc} % for calculating minipage widths
% Correct order of tables after \paragraph or \subparagraph
\usepackage{etoolbox}
\makeatletter
\patchcmd\longtable{\par}{\if@noskipsec\mbox{}\fi\par}{}{}
\makeatother
% Allow footnotes in longtable head/foot
\IfFileExists{footnotehyper.sty}{\usepackage{footnotehyper}}{\usepackage{footnote}}
\makesavenoteenv{longtable}
\usepackage{graphicx}
\makeatletter
\newsavebox\pandoc@box
\newcommand*\pandocbounded[1]{% scales image to fit in text height/width
  \sbox\pandoc@box{#1}%
  \Gscale@div\@tempa{\textheight}{\dimexpr\ht\pandoc@box+\dp\pandoc@box\relax}%
  \Gscale@div\@tempb{\linewidth}{\wd\pandoc@box}%
  \ifdim\@tempb\p@<\@tempa\p@\let\@tempa\@tempb\fi% select the smaller of both
  \ifdim\@tempa\p@<\p@\scalebox{\@tempa}{\usebox\pandoc@box}%
  \else\usebox{\pandoc@box}%
  \fi%
}
% Set default figure placement to htbp
\def\fps@figure{htbp}
\makeatother



\ifLuaTeX
\usepackage[bidi=basic]{babel}
\else
\usepackage[bidi=default]{babel}
\fi
% get rid of language-specific shorthands (see #6817):
\let\LanguageShortHands\languageshorthands
\def\languageshorthands#1{}


\setlength{\emergencystretch}{3em} % prevent overfull lines

\providecommand{\tightlist}{%
  \setlength{\itemsep}{0pt}\setlength{\parskip}{0pt}}



 
\usepackage[]{biblatex}
\addbibresource{references.bib}


\usepackage{cancel}
\KOMAoption{captions}{tableheading,figureheading}
\makeatletter
\@ifpackageloaded{tcolorbox}{}{\usepackage[skins,breakable]{tcolorbox}}
\@ifpackageloaded{fontawesome5}{}{\usepackage{fontawesome5}}
\definecolor{quarto-callout-color}{HTML}{909090}
\definecolor{quarto-callout-note-color}{HTML}{0758E5}
\definecolor{quarto-callout-important-color}{HTML}{CC1914}
\definecolor{quarto-callout-warning-color}{HTML}{EB9113}
\definecolor{quarto-callout-tip-color}{HTML}{00A047}
\definecolor{quarto-callout-caution-color}{HTML}{FC5300}
\definecolor{quarto-callout-color-frame}{HTML}{acacac}
\definecolor{quarto-callout-note-color-frame}{HTML}{4582ec}
\definecolor{quarto-callout-important-color-frame}{HTML}{d9534f}
\definecolor{quarto-callout-warning-color-frame}{HTML}{f0ad4e}
\definecolor{quarto-callout-tip-color-frame}{HTML}{02b875}
\definecolor{quarto-callout-caution-color-frame}{HTML}{fd7e14}
\makeatother
\makeatletter
\@ifpackageloaded{bookmark}{}{\usepackage{bookmark}}
\makeatother
\makeatletter
\@ifpackageloaded{caption}{}{\usepackage{caption}}
\AtBeginDocument{%
\ifdefined\contentsname
  \renewcommand*\contentsname{Índice}
\else
  \newcommand\contentsname{Índice}
\fi
\ifdefined\listfigurename
  \renewcommand*\listfigurename{Lista de Figuras}
\else
  \newcommand\listfigurename{Lista de Figuras}
\fi
\ifdefined\listtablename
  \renewcommand*\listtablename{Lista de Tabelas}
\else
  \newcommand\listtablename{Lista de Tabelas}
\fi
\ifdefined\figurename
  \renewcommand*\figurename{Figura}
\else
  \newcommand\figurename{Figura}
\fi
\ifdefined\tablename
  \renewcommand*\tablename{Tabela}
\else
  \newcommand\tablename{Tabela}
\fi
}
\@ifpackageloaded{float}{}{\usepackage{float}}
\floatstyle{ruled}
\@ifundefined{c@chapter}{\newfloat{codelisting}{h}{lop}}{\newfloat{codelisting}{h}{lop}[chapter]}
\floatname{codelisting}{Listagem}
\newcommand*\listoflistings{\listof{codelisting}{Lista de Listagens}}
\makeatother
\makeatletter
\makeatother
\makeatletter
\@ifpackageloaded{caption}{}{\usepackage{caption}}
\@ifpackageloaded{subcaption}{}{\usepackage{subcaption}}
\makeatother
\usepackage{bookmark}
\IfFileExists{xurl.sty}{\usepackage{xurl}}{} % add URL line breaks if available
\urlstyle{same}
\hypersetup{
  pdftitle={Anotações de Inferência Bayesiana},
  pdfauthor={Aulas de Luis G. Esteves \& Victor Fossaluza, digitadas por Gustavo S. Garone},
  pdflang={pt},
  colorlinks=true,
  linkcolor={blue},
  filecolor={Maroon},
  citecolor={Blue},
  urlcolor={Blue},
  pdfcreator={LaTeX via pandoc}}


\title{Anotações de Inferência Bayesiana}
\author{Aulas de Luis G. Esteves \& Victor Fossaluza, digitadas por
Gustavo S. Garone}
\date{2025-08-06}
\begin{document}
\maketitle

\renewcommand*\contentsname{Índice}
{
\hypersetup{linkcolor=}
\setcounter{tocdepth}{2}
\tableofcontents
}

\bookmarksetup{startatroot}

\chapter{Inferência Bayesiana}\label{inferuxeancia-bayesiana}

\part{Inferência Bayesiana}

\chapter{O Pensamento Bayesiano}\label{o-pensamento-bayesiano}

\section{Inferência}\label{inferuxeancia}

Podemos dizer que inferência é qualquer processo racional de redução de
\textbf{incerteza}. Se não há incerteza, não há necessidade da
estatística.

Como exemplo simples, considere uma caixa com 5 bolas, algumas verdes
(\(v\)), e o restante brancas (\(b\)). Suponha que amostramos duas
bolas, sem reposição. Isto é, retiramos duas bolas desta caixa, e que
ambas as bolas retiradas foram verdes. A partir deste experimento,
podemos \emph{inferir} que \(v \geq 2\), ou seja, ao menos duas bolas da
caixa são verdes. Caso retirássemos uma branca e uma verde, poderíamos
naturalmente inferir que ao menos uma bola da caixa é branca, e ao menos
uma é verde.

Note que, mesmo após esse processo, a incerteza sobre a composição da
caixa contínua. Contudo, com a nova informação, ocorre uma
\emph{atualização da incerteza}.

A inferência estatística é, portanto, o processo de redução de
incertezas a partir de dados e métodos estatísticos. Neste contexto, a
\emph{Inferência Bayesiana} é a inferência estatística baseada na
perspectiva \emph{subjetiva} de probabilidade.

\begin{tcolorbox}[enhanced jigsaw, colframe=quarto-callout-note-color-frame, bottomrule=.15mm, toptitle=1mm, left=2mm, toprule=.15mm, bottomtitle=1mm, colback=white, rightrule=.15mm, titlerule=0mm, opacityback=0, coltitle=black, opacitybacktitle=0.6, arc=.35mm, leftrule=.75mm, colbacktitle=quarto-callout-note-color!10!white, breakable, title=\textcolor{quarto-callout-note-color}{\faInfo}\hspace{0.5em}{Probabilidade}]

O objeto probabilidade pode ser estudada de diversas perspectivas.
Estamos acostumados com o \emph{cálculo de probabilidade}, em que
desenvolvemos teoremas e outros resultados a partir dos axiomas de
Kolmogorov.

\[
\begin{aligned}
P(A) \geq 0, \forall A \in \mathcal{F} \\
P(\Omega) = 1 \\
P\left(\bigcup\limits_{n=1}^\infty A_n\right) = \sum^\infty_{n=1} A_n
\end{aligned}
\]

Por outro lado, há pessoas que se ocupam em entender a probabilidade do
ponto de vista da \emph{Teoria da Medida}, preocupando-se em estudar a
probabilidade como uma função de medida.

Ainda assim, existe uma visão mais ``filosófica'' do estudo da
probabilidade, que se preocupa na \emph{interpretação} da probabilidade.
Uma interpretação
\href{https://inferencia-classica.netlify.app}{clássica} é da
probabilidade como medida de frequência, enquanto a interpretação
subjetiva da probabilidade, utilizada na inferência Bayesiana, toma a
probabilidade como uma medida de \emph{incerteza}.

Por exemplo, num lançamento de moedas, a teoria frequentista afirma que,
ao lançar uma moeda honesta um número grande de vezes, esperaria que
50\% dos lançamentos resultasse em cara. Por outro lado, na teoria
subjetiva da probabilidade, alguém que acredita que a moeda é honesta
diria, antes do primeiro lançamento desta, que não há razão de acreditar
que a moeda vá provavelmente cair cara, tão quanto acreditar que mais
provavelmente cairá coroa. Nesta visão, a probabilidade é um número que
\emph{mede, quantifica ou representa a incerteza} do observador sobre um
fenômeno.

Esta discussão sobre o significado de probabilidade é denominado
\emph{Teoria da Probabilidade}.

\end{tcolorbox}

\subsection{Cenários de Incerteza}\label{cenuxe1rios-de-incerteza}

Além do exemplo da caixa com bolas que discutimos anteriormente, podemos
pensar em outros cenários de incerteza:

Considere um indivíduo com erupções na pele. Este tipo de erupções pode
ser causado por doenças diversas, dentre elas, uma é considerada grave,
provocando incerteza no indivíduo sobre sua saúde. Sabendo disto, o
indivíduo vai ao hospital em busca de informações sobre sua condição. A
partir de experimentos, como exames médicos, a incerteza do indivíduo
sobre a doença e sua saúde é reduzida pelas informações obtidas.

Em outro exemplo, considere que um avião caiu num corpo d'água e sua
localização exata é desconhecida. Conforme descobrimos informações do
avião, como partes encontradas no mar, consideração de oceanográficos e
climáticos, análise técnica do modelo do avião, resultados de buscas
passadas e outras informações, podemos gradativamente reduzir a região
de queda do avião, até encontrarmos o ponto exato ou muito próximo desta
queda.

Este exemplo é mais real do que pensa! Técnicas bayesianas foram
utilizadas na
\href{https://www.technologyreview.com/2014/05/27/13283/how-statisticians-found-air-france-flight-447-two-years-after-it-crashed-into-atlantic/}{busca
pelo infame voo 447}

Num último exemplo, suponha que estamos interessados em estudar a
eficácia de um medicamento na redução da taxa do colesterol. A incerteza
jaz exatamente na eficácia do medicamento, isto é, quanto a taxa de
colesterol é reduziada após uso do medicamento. Para esta análise,
coletaríamos amostras aleatórias simples. Suponha, por hipótese, que o
colesterol dos pacientes em estudo amostrados
(\(\boldsymbol{X}_n = (X_1, \dots, X_n)\) seguem uma distribuição
Normal, com variância 16 e média desconhecida que queremos inferir:

\[
X \sim (?,16)
\]

No modelo clássico, usaríamos de métodos como o uso do estimador não
viesado de variância uniformemente mínima. Por outro lado, o bayesiano
começaria com uma incerteza antes da coleta da amostra e, com a
informação obtida, atualizaria sua incerteza em uma nova distribuição:
\[
\mathcal{D} \stackrel{\boldsymbol{X}_n}{\rightarrow} \mathcal{D} \lvert x_1,
\dots, x_n
\]

Esta ferramenta de atualização de informação é o cerne da abordagem
Bayesiana, e será extensivamente estudada pelo curso.


\printbibliography



\end{document}
